\section{Translation between \NAMD\ and X-PLOR configuration parameters}
\label{section:xplorequiv}

\NAMD\ was designed to provide many of the same molecular dynamics functions that
X-PLOR provides.  As such, there are many similarities between the types of parameters
that must be passed to both X-PLOR and \NAMD.  This section describes relations
between similar \NAMD\ and X-PLOR parameters.  

\begin{itemize}

\item
\XNCOMP{cutoff}{CTOFNB}
{When full electrostatics are not in use within \NAMD, 
these parameters have exactly the same meaning 
--- the distance at which electrostatic 
and van der Waals forces are truncated.  
When full electrostatics are in use
within \NAMD, the meaning is still very similar.  
The van der Waals force is still truncated at the specified distance, 
and the electrostatic force is still computed at every timestep 
for interactions within the specified distance.  
However, the \NAMD\ integration uses multiple time stepping to 
compute electrostatic force interactions beyond this distance 
every {\tt stepspercycle} timesteps.}

\item
\XNCOMP{vdwswitchdist}{CTONNB}
{Distance at which the van der Waals switching function becomes active.}

\item 
\XNCOMP{pairlistdist}{CUTNb}
{Distance within which interaction pairs will be included in pairlist.}

\item
\XNCOMP{1-4scaling}{E14Fac}{Scaling factor for 1-4 pair electrostatic interactions.}

\item
\XNCOMP{dielectric}{EPS}{Dielectric constant.}

\item
\XNCOMP{exclude}{NBXMod}
{Both parameters specify which atom pairs 
to exclude from non-bonded interactions.  
The ability to ignore explicit exclusions is {\it not} present within \NAMD, 
thus only positive values of {\tt NBXMod} have \NAMD\ equivalents.  
These equivalences are
\begin{itemize}
\item
{\tt NBXMod=1} is equivalent to {\tt exclude=none} 
--- no atom pairs excluded, 
\item
{\tt NBXMod=2} is equivalent to {\tt exclude=1-2} 
--- only 1-2 pairs excluded, 
\item
{\tt NBXMod=3} is equivalent to {\tt exclude=1-3} 
--- 1-2 and 1-3 pairs excluded, 
\item
{\tt NBXMod=4} is equivalent to {\tt exclude=1-4} 
--- 1-2, 1-3, and 1-4 pairs excluded, 
\item
{\tt NBXMod}=5 is equivalent to {\tt exclude=scaled1-4} 
--- 1-2 and 1-3 pairs excluded, 1-4 pairs modified. 
\end{itemize}}

\item
\XNCOMP{switching}{SHIFt, VSWItch, and TRUNcation}
{Activating the \NAMD\ option {\tt switching} is equivalent 
to using the X-PLOR options {\tt SHIFt} and {\tt VSWItch}.  
Deactivating {\tt switching} is equivalent 
to using the X-PLOR option {\tt TRUNcation}.}

\item
\XNCOMP{temperature}{FIRSttemp}
{Initial temperature for the system.}

\item
\XNCOMP{rescaleFreq}{IEQFrq}
{Number of timesteps between velocity rescaling.}

\item
\XNCOMP{rescaleTemp}{FINAltemp}
{Temperature to which velocities are rescaled.}

\item
\XNCOMP{restartname}{SAVE}{Filename prefix for the restart files.}

\item
\XNCOMP{restartfreq}{ISVFrq}
{Number of timesteps between the generation of restart files.}

\item
\XNCOMP{DCDfile}{TRAJectory}
{Filename for the position trajectory file.} 

\item
\XNCOMP{DCDfreq}{NSAVC}
{Number of timesteps between writing coordinates to the trajectory file.}

\item
\XNCOMP{velDCDfile}{VELOcity}
{Filename for the velocity trajectory file.}

\item
\XNCOMP{velDCDfreq}{NSAVV}
{Number of timesteps between writing velocities to the trajectory file.} 

\item
\XNCOMP{numsteps}{NSTEp}
{Number of simulation timesteps to perform.}

\end{itemize}

