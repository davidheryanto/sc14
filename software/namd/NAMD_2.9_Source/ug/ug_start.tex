
\section{Getting Started}
\label{section:start}

\subsection{What is needed}

Before running \NAMD, explained in section \ref{section:run}, 
the following are be needed:
\begin{itemize}
\item A CHARMM force field in either CHARMM or X-PLOR format.
\item
An X-PLOR format PSF file describing the molecular structure.
\item
The initial coordinates of the molecular system in the form of a PDB file.  
\item
A \NAMD\ configuration file.
\end{itemize}

\NAMD\ provides the \verb#psfgen# utility,
documented in Section~\ref{section:psfgen},
which is capable of generating the required PSF and PDB files by
merging PDB files and guessing coordinates for missing atoms.
If \verb#psfgen# is insufficient for your system,
we recommend that you obtain access to either CHARMM or X-PLOR,
both of which are capable of generating the required files.

\subsection{\NAMD\ configuration file}
\label{section:config}

Besides these input and output files, \NAMD\ also uses 
a file referred to as the {\it configuration file\/}.  
This file specifies what dynamics options and values that 
\NAMD\ should use, such as the number of timesteps to perform, 
initial temperature, etc.  
The options and values in this file control how 
the system will be simulated.  

A \NAMD\ configuration file contains a set of options and values.  
The options and values specified determine the exact behavior of
\NAMD, what features are active or inactive, how long the simulation
should continue, etc.  Section \ref{section:configsyntax} describes how
options are specified within a \NAMD\ configuration file.  Section
\ref{section:requiredparams} lists the parameters which are required
to run a basic simulation.  Section \ref{section:xplorequiv}
describes the relation between specific \NAMD\ and X-PLOR dynamics
options.  Several sample \NAMD\ configuration files are shown
in section \ref{section:sample}.


\subsubsection{Configuration parameter syntax}
\label{section:configsyntax}
Each line
in the configuration files consists of a $keyword$ identifying the option
being specified, and a $value$ which is a parameter to be used for this
option.  The keyword and value can be separated by only white space:
\begin{verbatim}
keyword            value
\end{verbatim}
or the keyword and value can be separated by an equal sign and white space:
\begin{verbatim}
keyword      =     value
\end{verbatim}
Blank lines in the configuration file are ignored.  Comments are prefaced by
a \verb!#! and may appear on the end of a line with actual values:
\begin{verbatim}
keyword            value          #  This is a comment
\end{verbatim}
or may be at the beginning of a line:
\begin{verbatim}
#  This entire line is a comment . . . 
\end{verbatim}
Some keywords require several lines of data.
These are generally implemented to either allow the data to be read from a file:
\begin{verbatim}
keyword            filename
\end{verbatim}
or to be included inline using Tcl-style braces:
\begin{verbatim}
keyword {
  lots of data
}
\end{verbatim}

The specification of the keywords is case insensitive 
so that any combination of 
upper and lower case letters will have the same meaning.  
Hence, {\tt DCDfile} and {\tt dcdfile} 
are equivalent.  The capitalization in the values, however, may be important.
Some values indicate file names, in which capitalization is critical.  
Other values such as {\tt on} or {\tt off} are case insensitive.

\subsubsection{Tcl scripting interface and features}
\label{section:tclscripting}

When compiled with Tcl (all released binaries) the config file
is parsed by Tcl in a fully backwards compatible manner with the
added bonus that any Tcl command may also be used.  This alone allows:
\begin{itemize}
 \item the ``\icommand{source}'' command to include other files (works w/o Tcl too!),
 \item the ``\icommand{print}'' command to display messages (``puts'' is broken, sorry),
 \item environment variables through the env array (``\$env(USER)''), and
 \item user-defined variables (``set base sim23'', ``dcdfile \${base}.dcd'').
\end{itemize}

Additional features include:
\begin{itemize}
 \item The ``\icommand{callback}'' command takes a 2-parameter Tcl procedure which
    is then called with a list of labels and a list of values during
    every timestep, allowing analysis, formatting, whatever.
 \item The ``\icommand{run}'' command takes a number of steps to run (overriding the
    now optional numsteps parameter, which defaults to 0) and can be
    called repeatedly.  You can ``run 0'' just to get energies.
 \item The ``\icommand{minimize}'' command is similar to ``run'' and performs
    minimization for the specified number of force evaluations.
 \item The ``\icommand{output}'' command takes an output file basename and causes
    .coor, .vel, and .xsc files to be written with that name.
    Alternatively, ``\icommand{output withforces}'' and
    ``\icommand{output onlyforces}'' will write a .force file
    either in addition to or instead of the regular files.
 \item Between ``run'' commands the \iparam{reassignTemp},
    \iparam{rescaleTemp}, and
    \iparam{langevinTemp} parameters can be changed to allow simulated
    annealing protocols within a single config file.
    The \iparam{useGroupPressure}, \iparam{useFlexibleCell},
    \iparam{useConstantArea}, \iparam{useConstantRatio},
    \iparam{LangevinPiston}, \iparam{LangevinPistonTarget},
    \iparam{LangevinPistonPeriod}, \iparam{LangevinPistonDecay},
    \iparam{LangevinPistonTemp}, \iparam{SurfaceTensionTarget},
    \iparam{BerendsenPressure}, \iparam{BerendsenPressureTarget},
    \iparam{BerendsenPressureCompressibility}, and
    \iparam{BerendsenPressureRelaxationTime}
    parameters may be changed to allow pressure equilibration.
    The \iparam{fixedAtoms}, \iparam{constraintScaling}, and
    \iparam{nonbondedScaling} parameters may be
    changed to preserve macromolecular conformation during minimization and
    equilibration (fixedAtoms may only be disabled, and requires that
    \iparam{fixedAtomsForces} is enabled to do this).
    The \iparam{consForceScaling} parameter may be changed to vary steering forces
    or to implement a time-varying electric field that affects specific atoms.
    The \iparam{eField}, \iparam{eFieldFreq}, and
    \iparam{eFieldPhase} parameters may be changed to implement
    at time-varying electric field that affects all atoms.
    The \iparam{alchLambda} and \iparam{alchLambda2}
    parameters may be changed during alchemical free energy runs.
 \item The ``\icommand{checkpoint}'' and ``revert'' commands (no arguments) allow
    a scripted simulation to save and restore to a prior state.
 \item The ``\icommand{exit}'' command writes output files and exits cleanly.
 \item The ``\icommand{abort}'' command concatenates its arguments into
    an error message and exits immediately without writing output files.
 \item The ``\icommand{numPes}'', ``\icommand{numNodes}'', and
    ``\icommand{numPhysicalNodes}'' commands allow performance-tuning
    parameters to be set based on the parallel execution environment.
 \item The ``\icommand{reinitvels}'' command reinitializes velocities to a
    random distribution based on the given temperature.
 \item The ``\icommand{rescalevels}'' command rescales velocities by
    the given factor.
 \item The ``\icommand{reloadCharges}'' command reads new atomic charges from
    the given file, which should contain one number for each atom, separated
    by spaces and/or line breaks.
 \item The ``\icommand{consForceConfig}'' command takes a list of
    0-based atom indices and a list of forces which replace the existing
    set of constant forces (\iparam{constantForce} must be on).
 \item The ``\icommand{measure}'' command allows user-programmed calculations to
    be executed in order to facilitate automated methods.  (For
    example, to revert or change a parameter.)  A number of measure
    commands are included in the NAMD binary; the module has been designed
    to make it easy for users to add additional measure commands.  
 \item The ``\icommand{coorfile}'' command allows NAMD to perform force and energy
    analysis on trajectory files.  ``coorfile open dcd {\tt filename}'' opens
    the specified DCD file for reading.  ``coorfile read'' reads the next
    frame in the opened DCD file, replacing NAMD's atom coordinates with the
    coordinates in the frame, and returns 0 if successful or -1 if  
    end-of-file was reached.  ``coorfile skip'' skips past one frame in the
    DCD file; this is significantly faster than reading coordinates and 
    throwing them away.  ``coorfile close'' closes the file.   
    The ``coorfile'' command is not available on the Cray T3E.

    Force and energy analysis are especially useful in the context of 
    pair interaction calculations; see Sec.~\ref{section:pairinteraction}
    for details, as well as the example scripts in Sec.~\ref{section:sample}.
\end{itemize}

Please note that while NAMD has traditionally allowed comments to be
started by a \# appearing anywhere on a line, Tcl only allows comments
to appear where a new statement could begin.  With Tcl config file
parsing enabled (all shipped binaries) both NAMD and Tcl comments are
allowed before the first ``run'' command.  At this point only pure Tcl
syntax is allowed.  In addition, the ``;\#'' idiom for Tcl comments will
only work with Tcl enabled.  NAMD has also traditionally allowed
parameters to be specified as ``param=value''.  This is supported, but
only before the first ``run'' command.  Some examples:

\begin{verbatim}
# this is my config file                            <- OK
reassignFreq 100 ; # how often to reset velocities  <- only w/ Tcl
reassignTemp 20 # temp to reset velocities to       <- OK before "run"
run 1000                                            <- now Tcl only
reassignTemp 40 ; # temp to reset velocities to     <- ";" is required
\end{verbatim}

NAMD has also traditionally allowed parameters to be specified as
``param=value'' as well as ``param value''.  This is supported, but only
before the first ``run'' command.  For an easy life, use ``param value''.

\subsubsection{Required \NAMD\ configuration parameters}
\label{section:requiredparams}

The following parameters are {\em required} for every
\NAMD\ simulation:

\begin{itemize}

\item
{\tt numsteps} (page \pageref{param:numsteps}),

\item
{\tt coordinates} (page \pageref{param:coordinates}),

\item
{\tt structure} (page \pageref{param:structure}),

\item
{\tt parameters} (page \pageref{param:parameters}),

\item
{\tt exclude} (page \pageref{param:exclude}), 

\item
{\tt outputname} (page \pageref{param:outputname}), 

\item
one of the following three:
\begin{itemize}
\item
{\tt temperature} (page \pageref{param:temperature}),

\item
{\tt velocities} (page \pageref{param:velocities}),

\item
{\tt binvelocities} (page \pageref{param:binvelocities}).
\end{itemize}

\end{itemize}

\noindent These required parameters specify the most basic properties of
the simulation.  %  that is to be performed.
In addition, it is highly recommended that 
{\tt pairlistdist} be specified with a 
value at least one greater than {\tt cutoff}.

